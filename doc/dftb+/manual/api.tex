\chapter{\dftbp{} API}

You can compile \dftbp{} into a library and access some of its functionality via
an API. Currently the API offers high level access only: you can set the current
geometry and extract energy and forces for that geometry.

\section{Building the library}

In order to compile the \dftbp{} library, issue
\begin{verbatim}
make api
\end{verbatim}
After the compilation, the library (\verb|libdftb+.a|) can be found in the
\verb|api/mm| directory of your build tree. You will find also all mod-files
here.

You can test the library functionality by issuing
\begin{verbatim}
make test_api
\end{verbatim}


\section{General guidelines}

Although the DFTB+ library contains nearly all internal routines of the DFTB+
code, you should access the code functionality only via the provided API and not
by calling those internal routines directly. We put great efforts in keeping the
API stable over time, while the interface of the internal routines can change
without notice. The API version can be found in the \verb|API_VERSION| file in
the \verb|api/mm| folder. We use semantic versioning, a change in the major
(first) version number indicates backwards incompatible changes, while changes
in the minor (second) version number indicate backwards compatible extensions.

The interface is documented in the source code file
\verb|api/mm/mmapi.F90|.
