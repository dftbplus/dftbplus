\chapter{Introduction}

The code {\dftbp} is the Fortran 2003 successor of the old {\dftb}
code, which implements the density functional based tight binding
approach~\cite{frauenheim-JPCM-14-3015}. The code has been completely
rewritten from scratch and extended with various features. The most
recent features of the code are described in
Ref.~\cite{dftbp-2020paper}.

The main features of {\dftbp} are:
\begin{itemize}
\item Non-scc and scc calculations (with an expanded range of SCC
  accelerators)
  \begin{itemize}
  \item Cluster/molecular systems
  \item Periodic systems (arbitrary $k$-point sampling, band structure
    calculations, etc.)
  \item Open boundary conditions (wire and semi-infinite contacts)
  \end{itemize}
\item l-shell resolved calculations possible
\item Spin polarised calculations (both collinear and non-collinear
  spin)
\item Geometry and lattice optimisation
\item Geometry optimisation with constraints (in xyz-coordinates)
\item Molecular dynamics (NVE, NPH, NVT and NPT ensembles as well as
  metadynamics)
\item Numerical vibrational mode calculations
\item Finite temperature calculations
\item Dispersion corrections (van der Waals interactions)
\item Ability to treat $f$-electrons
\item Linear response excited state calculations for cluster/molecular systems
\item Geometry optimisation and molecular dynamics in singlet and triplet
  excited states of spin-free molecules
\item LDA+U/pSIC extensions
\item $L \cdot S$ coupling
\item 3rd order on-site corrections (improved H-bonding)
\item Long range hybrid corrections
\item REKS calculations for a strongly correlated system
\item QM/MM coupling with external point charges (smoothing possible)
\item GPU accelleration and OpenMP and MPI parallelisation with a
  range of electronic solvers
\item Electronic transport calculations (non-equilibrium Green function
  methodology, transmission calculations)
\item More general electrostatic boundary conditions via a Poisson equation
  electrostatic solver
\item Automatic code validation (autotest system)
\item New user friendly, extensible input format (HSD)
\item Dynamic memory allocation
\item Additional tool for generating cube files for charge distribution,
  molecular orbitals, etc. (Waveplot)
\item Excitation energies for cluster/molecular systems using the particle-particle Random Phase Approximation.   
\end{itemize}
