
\chapter{Description of some relevant file formats}
\label{app:files}

\section{The XYZ File Format}

The \emph{XYZ file format} is a minimalist, plain-text representation
of molecular geometries widely used in quantum chemistry, molecular
mechanics, and atomistic simulation. It encodes atomic identities and
Cartesian coordinates in a fixed, human-readable layout without
explicit bonding or connectivity information.

The XYZ format was never formalized by a standards
organization. Instead, it evolved organically, with minor variations
appearing in different codes. This standardization vacuum led to
multiple incompatible extensions, particularly with respect to the
number and interpretation of columns in the atomic data section.

\subsection{Canonical Structure}

An XYZ file describing a system of $N$ atoms has the following
structure:

\begin{enumerate}
  \item The first line contains the integer number of atoms $N$.
  \item The second line is a comment line containing arbitrary text.
  \item The following $N$ lines contain atomic data.
\end{enumerate}

The comment line is frequently left empty but is syntactically
required by most readers. Some implementations overload it with
metadata such as energies or timestep data.

\subsection{Four-Column XYZ Format}

The most common and widely supported variant is the four-column XYZ
format.  Each atomic line consists of the atomic symbol followed by
three Cartesian coordinates, typically in \AA{}:
\[
\text{symbol}_i \quad x_i \quad y_i \quad z_i.
\]

A typical example is
\begin{verbatim}
3
Water molecule
O   0.000000   0.000000   0.000000
H   0.758602   0.000000   0.504284
H  -0.758602   0.000000   0.504284
\end{verbatim}

Here, atomic identity is specified by the chemical symbol
(case-sensitive by convention), and no information regarding bonding,
charge, or electronic state is encoded.

\subsection{Five-Column XYZ Format}

A five-column XYZ file extends the atomic line with an additional
per-atom quantity:
\[
\text{symbol}_i \quad x_i \quad y_i \quad z_i \quad q_i.
\]

The fifth column is most commonly interpreted as a partial atomic
charge. Because the meaning of the fifth column is not standardized,
files using this extension are not reliably portable across programs.

\subsection{Seven-Column XYZ Format}

The seven-column XYZ format is most frequently encountered in
molecular dynamics contexts. In this variant, three additional columns
are appended to store a Cartesian vector quantity:
\[
\text{symbol}_i \quad x_i \quad y_i \quad z_i \quad v_{x,i} \quad
v_{y,i} \quad v_{z,i}.
\]

These quantities are most often atomic velocities, though forces or
other vectors can also be used in practice.

A representative example is
\begin{verbatim}
2
Diatomic molecule with velocities
H   0.000000   0.000000   0.000000   0.010   0.000   0.000
H   0.740000   0.000000   0.000000  -0.010   0.000   0.000
\end{verbatim}

As with the five-column case, interpretation of the extra columns
depends entirely on external conventions.

\subsection{Use of the Comment Line}

The second line of an XYZ file is commonly overloaded with additional
metadata, such as total energies or simulation times. For example,
\begin{verbatim}
5
Energy = -152.347839 Ha
\end{verbatim}

While convenient, this practice further emphasizes the informal nature
of the format, as such metadata are not machine-interpretable without
prior agreement.

\section{XYZ Formats in DFTB+}

\subsection{Introduction}

DFTB+ supports the XYZ family of formats primarily as a \emph{geometry exchange
and output format}, rather than as its most feature-complete internal input
representation. While the DFTB+ native geometry format (\texttt{.gen}) provides
explicit support for atom types, lattice vectors, and periodicity, XYZ files are
supported for both input and output in several distinct variants.

These variants differ in the number of columns and in the interpretation of the
comment line, depending on the mode of operation (single-point calculation,
geometry optimization, or molecular dynamics).

\subsection{XYZ as an Input Geometry}

\subsubsection{Four-Column XYZ (Input Mode)}

DFTB+ accepts standard four-column XYZ files as geometry input when
the \texttt{Geometry = XYZ \{ ... \}} block is used. In this mode,
each atomic line has the form
\[
\text{symbol}_i \quad x_i \quad y_i \quad z_i,
\]
with Cartesian coordinates given in \AA{}.

The atomic symbol is mapped internally to a DFTB+ \emph{species} via
the \texttt{Species} definition block in the input file. No charge,
velocity, or constraint information can be read from an XYZ input.

Periodic boundary conditions cannot be specified in pure XYZ input;
systems read in this way are treated as non-periodic unless lattice
information is provided elsewhere in the input.

\subsection{XYZ Output: Final Geometry}

\subsubsection{Four-Column XYZ (\texttt{geo\_end.xyz})}

At the end of a calculation, DFTB+ can write the final geometry to an
XYZ file, usually named \texttt{geo\_end.xyz}. For non-self-consistent
calculations, this file uses the conventional four-column format:
\[
\text{symbol}_i \quad x_i \quad y_i \quad z_i.
\]

No velocities or forces are included in the \texttt{geo\_end.xyz}.

\subsection{XYZ Output: Geometry Optimization Trajectories}

\subsubsection{Multi-Frame Four-Column XYZ}

During geometry optimization, DFTB+ can write intermediate structures
using the XYZ format. Each optimization step is optionally written as
a separate XYZ ``frame'' in a single file.  This produces a
concatenated XYZ trajectory, compatible with common molecular
visualization tools.

\subsection{XYZ Output: Molecular Dynamics}

\subsubsection{Seven-Column XYZ (MD Trajectories)}

In molecular dynamics simulations, DFTB+ can write extended XYZ
trajectory files that include atomic velocities. In this mode, each
atomic line contains seven columns:
\[
\text{symbol}_i \quad x_i \quad y_i \quad z_i
\quad v_{x,i} \quad v_{y,i} \quad v_{z,i}.
\]

Coordinates are given in \AA{}, and velocities are given in \AA{}/ps.

MD steps can write an XYZ frame with:
\begin{itemize}
  \item the number of atoms,
  \item a comment line (often including the MD step or simulation time),
  \item seven-column atomic data.
\end{itemize}
