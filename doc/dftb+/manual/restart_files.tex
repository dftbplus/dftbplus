
\chapter{Description of restart files}
\label{app:restartfiles}

\subsection{charges.bin / charges.dat}

Initial charges and the current orbital charges are stored in these files. Both
contain the same information, but \verb|charges.bin| is stored as unformatted
binary data, while \verb|charges.dat| is a text file.

The first line of the file is:
\begin{verbatim}
  version tBlockCharges tImaginaryBlock nSpin CheckSum
\end{verbatim}

Where version is currently 3, tBlockCharges and tImaginaryBlock are logical
variables as to whether real and imaginary block charges are present. nSpin is
the number of spin channels (1, 2 or 4 for spin free, collinear and
non-collinear) and checksum is the totals for the charges in each spin channel.

The subsequent nAtom $\times$ nSpin lines contain the individual orbital
occupations for each atom in spin channel 1 (then 2 $\ldots$ 4, if present).

If tBlockCharges is true, then the on-site block charges for each atom and spin
channel are stored, followed by the imaginary part if tImaginaryBlock is true.

Examples of the contents of \verb|charges.dat| are given below for an H$_2$O
molecule in the $yz$ aligned with its dipole along $y$.  Using the mio-1-1
Slater-Koster set, this file would contain:
\begin{tiny}
\begin{verbatim}
           3 F F           1   8.0000000000000018
   6.5926151655316767        0.0000000000000000        0.0000000000000000        0.0000000000000000
  0.70369241723366482
  0.70369241723466003
\end{verbatim}
\end{tiny}

When \is{OrbitalResolved = No}. So, this is version 3 of the format, without
block charges and it is spin free with 8 electrons. The electronic charges are
grouped into the lowest atomic orbitals of each atom in this case. There is some
small numerical noise in some of these these values ($<10^{-14}$).

With \is{OrbitalResolved = Yes}, the oxygen has 1.7 $2s$ electrons and 4.83 $2p$
orbitals (electrons listed in the lowest labelled state in each shell).
\begin{tiny}
\begin{verbatim}
           3 F F           1   8.0000000000000018
   1.7335403452609417        4.8346073382345685        0.0000000000000000        0.0000000000000000
  0.71592615825295036
  0.71592615825154060
\end{verbatim}
\end{tiny}

While for a pseudo-SIC calculation, where the net spin is 0:
\begin{tiny}
\begin{verbatim}
           3 T F           2   8.0000000000000018        0.0000000000000000
   1.7566193972978825        1.7147230821039328        1.2018994732683752        2.0000000000000013
  0.66337902366501833
  0.66337902366479040
   0.0000000000000000        0.0000000000000000        0.0000000000000000        0.0000000000000000
   0.0000000000000000
   0.0000000000000000
   1.7566193972978825      -0.28455491415632111        7.9592308124161928E-014  -7.7482799007142168E-027
 -0.28455491415632111        1.7147230821039328        5.6922736516833719E-014   4.1776281206242259E-026
   7.9592308124161928E-014   5.6922736516833719E-014   1.2018994732683752       -4.6749196043609904E-016
  -7.7482799007142168E-027   4.1776281206242259E-026  -4.6749196043609904E-016   2.0000000000000013
  0.66337902366501833
  0.66337902366479040
   0.0000000000000000        0.0000000000000000        0.0000000000000000        0.0000000000000000
   0.0000000000000000        0.0000000000000000        0.0000000000000000        0.0000000000000000
   0.0000000000000000        0.0000000000000000        0.0000000000000000        0.0000000000000000
   0.0000000000000000        0.0000000000000000        0.0000000000000000        0.0000000000000000
   0.0000000000000000
   0.0000000000000000
\end{verbatim}
\end{tiny}
Note the 0 blocks for the spin polarisation in channel 2, and also that the
diagonal of the block charges matches the orbital charges for the atoms.

\subsection{contact.bin / contact.dat}
\label{app:contactfiles}

Self-consistent transport calculations require contact potential shifts. The
format of the \verb|shiftcont_*| files are either ascii (.dat) or binary
(.bin). The \verb|shiftcont_*.dat| files have the following format:

The first two lines of the file are:
\begin{verbatim}
  version
  nContactAtoms maxShells spinChannels tBlockCharges
\end{verbatim}

\begin{itemize}
\item version: The file format is currently version 1
\item nContactAtoms: Number of atoms in the contact
\item maxShells: Maximum number of angular shells on the contact atoms
\item spinChannels: Number of spin channels in the system (1 for spin free, 2
  for spin polarized)
\item tBlockCharges: a logical flag (T/F) as to whether block charges are
  present in the file (these are required for +U calculations).
\end{itemize}
This is then followed by lines for
\begin{itemize}
\item The number of orbitals on the atoms
\item Shifts for the shells of the atoms
\item Charges for individual orbitals
\end{itemize}

If tBlockCharges is true, the block shifts and charges are then given for each
spin channel and atom.

Finally the Fermi level(s) for the contact are printed (this can be over-ridden
in the input at calculation time, see the \is{FermiLevel} keyword in section
\ref{kw:transport.FermiLevel}).

An earlier format for contacts is also supported. This lacks the first line
containing the the version number, along with the logical flag and sections
relating to block charges.
